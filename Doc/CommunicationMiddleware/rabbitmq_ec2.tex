% TCM: TOTEM Communication Middleware
% Copyright: Copyright (C) 2009-2012
% Contact: denis.conan@telecom-sudparis.eu, michel.simatic@telecom-sudparis.eu
% Permission is granted to copy, distribute and/or modify this document
% under the terms of the GNU Free Documentation License, Version 1.3
% or any later version published by the Free Software Foundation;
% with no Invariant Sections, no Front-Cover Texts, and no Back-Cover Texts.
% A copy of the license is included in the section entitled "GNU
% Free Documentation License".

\section{Installation of RabbitMQ and Pika on Amazon EC2}
\label{S_installation_rabbitmq_ec2}

 \note{Important note:}{This appendix is deprecated since we have not
   done recent tests on the Amazon Cloud Computing Platform
   EC2. Please ask if you need an update of this section.}

In this section, we explain how to launch and configure an EC2
instance for executing the RabbitMQ broker (in Erlang), and some Pika
producers and consumers (in Python). The steps presented in this
section could be used to create a dedicated EC2 ``Amazone Machine
Image'' (AMI), as named in the Amazone EC2 vocabulary, for TOTEM.
Similar instructions are included in a separate
section (namely Section~\ref{S_installation_rabbitmq_local}) for
installing RabbitMQ on a desktop computer.

\subsection{Create an ``Instance''}
\label{SS_ec2_create_instance}

First of all, create an account on Amazon Elastic Compute Cloud
(EC2): \begin{small}\texttt{https://\-aws.\-amazon.\-com}\end{small}. Then,
connect to the ``AWS Management Console'' and create an instance as follows:
\begin{enumerate}
\item Choose the \textsf{EC2} page.
\item Select an ``Amazon Machine Image'' by ``Viewing'' all images of
  ``All platforms'' with for example the identifier
  \texttt{ami-06f80e6f}. This image is an Ubuntu operating system
  version \textsf{10.4}.
\item Launch an image using the contextual menu of the AMI you
  previously selected (by right clicking). In the ``Request Instances
  Wizard'', choose an ``Instance type'' of type \textsf{Micro} for
  instance: it is sufficient for our tests. At the ``Create Key Pair''
  step, create a new key pair and do not forget to load it in order to
  be granted an access through a SSH connection. Be careful! If you
  loose this key, you somewhat loose the instance you have just
  created.  The saved file (\textsf{*.pem}) must be put in a directory
  with restricted access (usely named \textsf{\string~/.ssh} on a
  Unix-like system, and with the rights \textsf{0700}) and must be
  assigned restricted rights (\textsf{0400}). At the ``Configure
  Firewall'' step, create a new security group. At the end of this
  procedure, the instance is running: Go to see the ``Instances'' by
  selecting the entry \textsf{Instances} in the menu on the left.
\item Browse the contextual menu of the instance that is running and
  have a look at the ``Instance Management'' menu and the ``Instance
  Lifecycle'' menu.
\item When an instance is selected, information such as the ``Public
  DNS'' are provided in the frame under the list of instances. A
  ``Public DNS'' is an IP address such as for
  example \begin{small}\texttt{ec2-174-129-122-66.compute-1.amazonaws.com}\end{small}
\end{enumerate}

\subsection{Configure the ``Instance''}
\label{SS_ec2_configure_instance}

The next step is to connect to the instance and install the software
for the example application. This is done as follows:
\begin{enumerate}
\item For the instance you have created following the steps of the
  previous section, open the TCP port number \textsf{22} for the
  procotol SSH. Go to the ``Security Groups'' frame by selecting the
  corresponding entry in the menu on the left. Select the security
  group you have previously created and add a new ``Allowed
  Connection'' of ``Connection Method'' type \textsf{SSH} and
  ``Source'' value \textsf{IP\_address\_of\_your\_computer/0}. The
  other information such as ``From Port'' and ``To Port'' are
  correctly configured to the default value \textsf{22}.
\item You can now open a shell connection to the instance using the
  following command (by adapting it to your use case):
\begin{shellcmd}
\$ ssh -i \$\{KeyPairFile.pem\} bitnami@\$\{PublicDNS\}
\end{shellcmd}
Note that you need the ``Key Pair'' file and the ``Public DNS''. The
login name \textsf{bitnami} is a default login name provided by the
AMI providers. This user is a ``sudoer'' and can thus use the command
\texttt{sudo} for executing a command as the user \texttt{root} when
necessary.
\item Install the Ubuntu package for Erlang with all its
  dependencies using the utility tool \texttt{apt-get}:
\begin{shellcmd}
sudo apt-get install erlang
\end{shellcmd}
\item Install the software RabbitMQ, version 2.7.1, by using the
  following commands:\\
\begin{shellcmd}
\$ wget http://www.rabbitmq.com/releases/rabbitmq-server/v2.7.1/\textbackslash
  rabbitmq-server-generic-unix-2.7.1.tar.gz
\$ tar xfz rabbitmq-server-generic-unix-2.7.1.tar.gz
\end{shellcmd}
Note that we do not install the Ubuntu package of RabbitMQ, but rather
prefer installing the software from a regular Unix archive so that we
will execute RabbitMQ as a non-\texttt{root} user.  The broker of
RabbitMQ, called the server in RabbitMQ terminology, is now installed
in the directory \textsf{/home/bitnami/rabbitmq\_server-2.7.1}.  The
commands of RabbitMQ (\textsf{rabbitmq-server} to launch the broker
and \textsf{rabbitmqctl} to control the broker) are available in the
directory \textsf{/home/bitnami/rabbitmq\_server-2.7.1/sbin}. In order
to add the RabbitMQ commands to the shell path, and to configure the
location of the database and the log files, execute the following
commands:
\begin{shellcmd}
\$ echo "PATH=\${PATH}:rabbitmq\_server-2.7.1/sbin" >> .bashrc
\$ mkdir /home/bitnami/rabbitmq\_server-2.7.1/mnesia
\$ echo "export RABBITMQ\_MNESIA\_BASE=/home/bitnami/rabbitmq\_server-2.7.1/mnesia" >> .bashrc
\$ mkdir /home/bitnami/rabbitmq\_server-2.7.1/log
\$ echo "export RABBITMQ\_LOG\_BASE=/home/bitnami/rabbitmq\_server-2.7.1/log" >> .bashrc
\$ . .bashrc \# to update the environment of the current shell connection
\end{shellcmd}
Note that RabbitMQ environment variables default to the values
\textsf{/var/lib/rabbitmq/mnesia} and \textsf{/var/log/rabbitmq},
preventing the launching of the broker as a non-\texttt{root} user.
We terminate the installation of RabbitMQ by following the
instructions provided in the section ``Issues with hostname'' of the
Web page \textsf{http://www.rabbitmq.com/ec2.html} and we apply the
following commands:
\begin{shellcmd}
sudo -s
echo "rabbit" > /etc/hostname
echo "127.0.0.1 rabbit" >> /etc/hosts
hostname -F /etc/hostname
exit
\end{shellcmd}
\item Your RabbitMQ installation can be tested by launching the broker
  as follows:
\begin{shellcmd}
\$ rabbitmq-server -detached \# execute in the background
Activating RabbitMQ plugins ...
0 plugins activated:
\$ rabbitmqctl status
Status of node rabbit@rabbit ...
[{running_applications,[{rabbit,"RabbitMQ","2.7.1"},
                        {mnesia,"MNESIA  CXC 138 12","4.4.14"},
                        {os\_mon,"CPO  CXC 138 46","2.2.5"},
                        {sasl,"SASL  CXC 138 11","2.1.9.2"},
                        {stdlib,"ERTS  CXC 138 10","1.17"},
                        {kernel,"ERTS  CXC 138 10","2.14"}]},
 {nodes,[{disc,[rabbit@rabbit]}]},
 {running\_nodes,[rabbit@rabbit]}]
...done.
\$ rabbitmqctl stop
Stopping and halting node rabbit@rabbit ...
...done.
\$ rabbitmqctl status
Status of node rabbit@rabbit ...
Error: unable to connect to node rabbit@rabbit: nodedown
diagnostics:
- nodes and their ports on rabbit: [{rabbitmqctl2776,45667}]
- current node: rabbitmqctl2776@rabbit
- current node home dir: /home/bitnami
- current node cookie hash: 58UmUjslvHMdJuJyRDcEag==
\$ jobs
\end{shellcmd}
\item In order to allow communication with the RabbitMQ broker
  installed on the cloud from your computer, add a new ``Allowed
  Connection'' to your security group with the following parameters:
  ``Connection Method'' is \textsf{Custom}, ``Protocol'' is
  \textsf{TCP}, ``From Port'' and ``To Port'' are \textsf{5672}
  (default port number of RabbitMQ broker), ``Source IP'' is
  \textsf{IP\_address\_of\_your\_computer/0}.
\item Install the software Pika, version 0.9.5 by executing the
  following commands
\begin{shellcmd}
\$ wget http://pypi.python.org/packages/source/p/pika/pika-v0.9.5.tar.gz
\$ tar xfz pika-v0.9.5.tar.gz
\end{shellcmd}
Note that we do not install the Python packages using the \texttt{pip}
or \texttt{easy\_install} utility tools; you can thus perform this
step differently. The Python package Pika is installed in the
directories \textsf{/home/bitnami/pika-v0.9.5}. Insert this
package in the path of Python by adding the directories in the shell
variable \textsf{PYTHONPATH}:
\begin{shellcmd}
\$ echo "export PYTHONPATH=/home/bitnami/pika-v0.9.5" >> .bashrc
\end{shellcmd}
\item Check that Pika is correctly installed by trying the following
  Python commands that demonstrate that the Pika package can be
  imported in a Python script:
\begin{shellcmd}
\$ python
Python 2.6.5 (r265:79063, Apr 16 2010, 13:09:56) 
[GCC 4.4.3] on linux2
Type "help", "copyright", "credits" or "license" for more information.
>>> import pika
>>> quit()
\end{shellcmd}
\end{enumerate}

\subsection{Create and share a TOTEM ``Amazon Machine Image''}
\label{SS_ec2_create_share_ami}

To be decided if this is the role of the project. We know how to do it.

\endinput
