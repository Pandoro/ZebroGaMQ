% TCM: TOTEM Communication Middleware
% Copyright: Copyright (C) 2009-2012
% Contact: denis.conan@telecom-sudparis.eu, michel.simatic@telecom-sudparis.eu
% Permission is granted to copy, distribute and/or modify this document
% under the terms of the GNU Free Documentation License, Version 1.3
% or any later version published by the Free Software Foundation;
% with no Invariant Sections, no Front-Cover Texts, and no Back-Cover Texts.
% A copy of the license is included in the section entitled "GNU
% Free Documentation License".

\section{Requirements for the communication infrastructure}
\label{S_requirements}

The requirements listed in this section are the
requirements for the communication infrastructure with their corresponding 
status.

\begin{enumerate}
\item Messaging
\label{R_1}
\begin{enumerate}
\item Destination of message
\label{R_1_a}
\begin{enumerate}
\item Point-to-point: From one entity (game server, game master, gamer
  / player, or spectator) to any other entity which the sending entity
  is aware of. (Mandatory)
\label{R_1_a_i}
\begin{itemize}
\item \done{All the bindings (between exchanges and queues) and
  the routing keys of the messages include the identity of the sender
  and of the addressee or ``all'' (all the entities).}
\end{itemize}
\item Broadcast to game players: It is possible for one entity (game
  server, game master, gamer / player, or spectator) to broadcast to
  all the other entities (including itself) (Mandatory)
\label{R_1_a_ii}
\begin{itemize}
\item \done{Using ``all'' as the addressee in the routing key, a
  message is broadcast to all the entities of the game instance. Note
  that the sender also receives its broadcast message.}
\end{itemize}
\end{enumerate}
\item Type of messages
\label{R_1_b}
\begin{enumerate}
\item Notification: An entity is able to send (``fire and forget''
  mode) an information to other entity(ies). (Mandatory)
\label{R_1_b_i}
\begin{itemize}
\item \done{This is the messaging paradigm of AMQP\footnote{The
    middleware \textsf{RabbitMQ} used for communication during the
    game play conforms to the standard AMQP, Advanced Message Queuing
    Protocol. Both AMQP and \textsf{RabbitMQ} are introduced in
    Section~\ref{S_presentation_concepts}.}.}
\end{itemize}
\item Request/Response: When an entity sends such a message, it blocks
  itself until it receives a response to its request (RPC
  mode). (Mandatory)
\begin{itemize}
\item \done{This is the messaging paradigm of XML-RPC. The login
  exchanges are performed using RPC mode.}
\end{itemize}
\label{R_1_b_ii}
\item Request/Response with ``future'': When an entity sends such a
  message, it continues operation and receives a delivery receipt at a
  later stage. (Important)
\begin{itemize}
\item \done{Futures can be implemented during the game play. This feature is 
not demonstrated in the illustrative application. But ask for
  an example to communication infrastructure designers when needed.}
\end{itemize}
\label{R_1_b_iii}
\end{enumerate}
\item Message reliability: Each message has a ``reliability''
  attribute. (Important)
\label{R_1_c}
\begin{itemize}
\item If this attribute is set to true, message will arrive to all of
  its destinations (even in the case of a broadcast message), even
  though some mobiles crashed or are disconnected.  However, if a
  mobile crashes and recovers, or disconnects for more than a given
  number of minutes (this number is configured at game server level),
  messages will be lost (this is to prevent the broker and the mobiles
  from being overflowed by undelivered messages).
\begin{itemize}
\item \done{This corresponds to the ``persistent'' attribute of
  messages in AMQP specification. By default, messages are not
  persistent. This feature is not demonstrated in the illustrative application.
  But ask for an example to communication infrastructure
  designers when needed. In addition, AMQP queues can be durable, that
  is they survive a broker crash and restart.}
\end{itemize}
\item If this attribute is set to false, messages can be lost.
\begin{itemize}
\item \done{This is the default configuration of AMQP messages.}
\end{itemize}
\end{itemize}
\item Order of messages
\label{R_1_d}
\begin{enumerate}
\item FIFO: Messages originating from the same source are received in
  the order with which they were sent by all the recipients. (Mandatory)
\label{R_1_d_i}
\begin{itemize}
\item \done{\textsf{RabbitMQ} guarantees ordering of the messages, as
  long as the messages in question follow the exact same path through
  the clients and the broker (same publisher, same exchange, same
  queue and same consumer). Be careful! This remains true as long as
  the broker does not crash and recover.}
\end{itemize}
\end{enumerate}
\item Platform-independent message encoding: The communication
  middleware is able to take care of problems related to differences
  of data encoding on the different platforms participating to the
  communication (Endian problem, Strings encoding...). (Important)
\label{R_1_e}
\begin{itemize}
\item \done{\textsf{Rabbit} treats a message payload as an opaque
  binary. We use Strings coding.}
\end{itemize}
\item Message security: Messages can be
  encrypted in order to avoid the decoding of messages contents by
  third parties. (Nice to have)
\label{R_1_f}
\begin{itemize}
\item \issue{AMQP promotes the use of SASL, which hasn't been experimented yet.}
\end{itemize}
\end{enumerate}
\item Presence service: At any time, an entity (mobile or game server)
  knows which session members are accessible.
\label{R_2}
\begin{itemize}
\item \postponed
\end{itemize}
\item Disconnection management (Mandatory, risk Medium). In case of
  disconnection (triggered by the application or not) from the broker,
  a mobile (Reminder: we assume that the game server is always
  connected to the broker) must:
\label{R_3}
\begin{enumerate}
\item Be aware of disconnection from network
\label{R_3_a}
\begin{itemize}
\item \done{Libraries developed for Java J2SE, Android and JavaScript 
applications have a ``maximum number of retries'' property, which can be set by
the user. If a disconnection occurs during the consumption or the publication 
of messages, 
the library tries to reconnect, to consume and possibly to publish messages 
(every second) until the maximum number of retries value is reached. If the 
network stays 
unreachable after this time period, the application stops its reconnection 
routine and the messages intended to this application stay on the broker 
queues.}
\end{itemize}
\item Be aware of the disconnection from the broker
\label{R_3_b}
\begin{itemize}
\item \postponed
\end{itemize}
\item Have tools to go on communicating:
\label{R_3_c}
\begin{enumerate}
\item The communication middleware always accepts messages from the
  application which uses it, even if it cannot communicate with the
  other entities (e.g. other mobiles and server) participating to the
  protocol.

  Said in other words: Messages are queued locally on the mobile
  rather than being deleted because the mobile is not connected to the
  broker. When connection to the broker is back, these queued messages are
  sent.
\label{R_3_c_i}
\begin{itemize}
\item \postponed
\end{itemize}
\item Each message can be given a delivery timeout. If such a timeout
  is set and the message is not delivered to destination after such
  timeout, an exception is sent to the application. The application
  can invoke a primitive to cancel such a message.
\label{R_3_c_ii}
\begin{itemize}
\item \postponed
\end{itemize}
\end{enumerate}
\end{enumerate}
\item Can send small messages as well as large binary blobs (e.g. from
  Location Survey tool). (Important)
\label{R_4}
\begin{itemize}
\item \done{By introducing the concept of framing, AMQP messages are
  chopped up into frames for interleaving on transport, which is
  useful for instance for large messages.}
\end{itemize}
% Functionalities developed by Totem project on top of the
% communication middleware to facilitate multiplayer game development
\item Totem offers game session concept above the communication
  middleware. In particular, broadcasts are sent to all session
  members. (Important)
\label{R_5}
\begin{itemize}
\item \done{This is the role of this document to introduce for such
  concept of game session through the AMQP concepts of virtual host,
  exchange, queue, routing key and binding.}
\end{itemize}
% Access to all of the functionalities (provided by the communication
% middleware and developed on top of it) by an application
\item Integrates with web framework
\label{R_8}
\begin{enumerate}
\item The web framework is able to send/receive messages through/from
  the communication middleware. (Mandatory)
\label{R_6_a}
\begin{itemize}
\item \donenocomment
\end{itemize}
\item The code of the broker can be integrated within the code of the
  framework. (Nice to have)
\label{R_6_b}
\begin{itemize}
\item \donenocomment
\end{itemize}
\end{enumerate}
\item An (Android) mobile client is able to send/receive messages
  through/from the communication middleware. (Mandatory)
\label{R_7}
\begin{itemize}
\item \done{The integration of Android running mobile devices is
  demonstrated in the example application of
  Section~\ref{S_integration}.}
\end{itemize}
\item A client (mobile or server) can communicate with the
  communication middleware through HTTP protocol. This protocol is
  used to connect the MPEG player and the communication
  middleware. This protocol may also be used in the case Telecom
  operators do not allow direct socket connections from mobile to the
  Internet. In this case, the mobile has to use HTTP protocol instead
  of plain socket-based library. (Mandatory)
\label{R_8}
\begin{itemize}
\item \done{Libraries are provided to enable communication between JavaScript 
applications and the broker (through a proxy). The communication is made with 
HTTP via AJAX requests.}
\end{itemize}
\item Easy setup and deployment. (Mandatory)
\label{R_9}
\begin{itemize}
\item \done{This document includes several sections in the appendix
  describing installation procedures.}
\end{itemize}
\item Standards based (Important).
\label{R_10}
\begin{itemize}
\item \done{\textsf{RabbitMQ} conforms to the version 0.9.1 of AMQP
  standard.}
\end{itemize}
\item Broker is able to push messages to game server (Mandatory, risk
  zero)
\label{R_11}
\begin{itemize}
\item \done{AMQP protocol pushes messages as soon as the consumer asks
  for it. By dedicating a thread for consuming messages in a loop,
  messages will be pushed by the broker at the speed of consumer
  calls of the AMQP command \texttt{consume}.}
\end{itemize}
\item Broker is able to push messages to mobiles (Important as it
  lowers latency for delivering messages to mobiles. Nevertheless it
  is not mandatory, as game design can limit the visibility of an
  important latency.)
\label{R_12}
\begin{itemize}
\item \done{Same answer as for Requirement~\ref{R_11}.}
\end{itemize}
\item On the mobile, if multiple communication channels (Wifi, GPRS,
  etc.) are available
\label{R_13}
\begin{enumerate}
\item When the player launches a game, the game tries to connect
  to the broker: We rely on the mobile OS to choose (or give to the
  user the choice of) the communication channel to use to connect to
  the broker (Mandatory, risk Medium: On Android, we can use method
  setNetworkPreference of the ConnectivityManager but there is little
  documentation).
\label{R_13_a}
\begin{itemize}
\item \irrelevant{Not an issue for the communication middleware.}
\end{itemize}
\item In the case the mobile was connected to Wifi and the player gets
  out of the Wifi zone, the telephone reconnects automatically to
  another available network (Nice to have because the player is not
  aware that it is going to use her data plan now).
\label{R_13_b}
\begin{itemize}
\item \irrelevant{Not an issue for the communication middleware.}
\end{itemize}
\item We keep track of ``free'' Internet connections of the user
\label{R_13_c}
\begin{itemize}
\item \irrelevant{Not an issue for the communication middleware.}
\end{itemize}
\end{enumerate}
\item Integrates with Android Cloud to Device Messaging Framework
  (Nice to have, as it is Android specific.)
\label{R_14}
\begin{itemize}
\item \irrelevant{It is a feature which allows to send small out of
  application messages. It does not have to be integrated with
  \textsf{RabbitMQ}.}
\end{itemize}
% Performance
\item Latency. Communication middleware requires a maximum of 100
  milliseconds (100 is an estimation which should fit most of Totem
  games) to send a message from a mobile client to the game server (if
  the mobile is connected with the broker thanks to a Wifi network).
\label{R_15}
\begin{itemize}
\item \done{
In the following array, we present the average performances for the publication 
and the consumption of one message containing about 10 characters. 
Results are given in milliseconds. 
\begin{tabular}{|l|l|c|c|} \hline
Client & Network & Publish & Consume  \\ \hline
\multirow{2}{*}{Android application} & WiFi & 2.9 & 7.8 \\
& 3G+ & 1.6 & 120 \\ \hline
\multirow{2}{*}{JavaScript application} & WiFi & 10.9 & 111 \\
& 3G+ & 18.2 & 371 \\ \hline
\end{tabular}
\label{mw_performances}}

\end{itemize}
\item Throughput. Communication middleware is able to withstand at
  least 1 notification per mobile client per second per game server
  (the configuration of each game server is the configuration of a
  free Amazon server), with at least 500 mobile clients. Above 500
  clients (this 500 number is currently an estimation; A more precise
  value should be computed thanks to a marketing study!), we consider
  that the game editor has enough customers to be able to pay for the
  Cloud service.
\label{R_16}
\begin{itemize}
\item \issue{Confident, but not experimented because no performance
  tests, yet.}
\end{itemize}
% Miscellaneous
\item Identity, user management (ideally via Facebook Connect, OAuth,
  OpenID, etc.) (Mandatory). If the communication
  middleware requires user login functionality, this functionality
  database can be replaced by mechanisms like Facebook Connect, OAuth,
  OpenID, etc.
\label{R_17}
\begin{itemize}
\item \irrelevant{The TOTEM \textsf{Django} game server manages the
  log-ins and passwords. Thus, it is not an issue for the
  communication middleware.}
\end{itemize}
\item Cloud support: Amazon, or Google App Engine. (Mandatory)
\label{R_18}
\begin{itemize}
\item \done{The cloud integration is not demonstrated with Google
  App. Engine (since the latter provides its own software
  infrastructure), but with Amazon
  EC2. Cf.~Section~\ref{S_installation_rabbitmq_ec2} of the appendix
  for the installation procedure.}
\end{itemize}
\end{enumerate}

\endinput
